\magnification\magstep1
\baselineskip14pt
\parskip5pt
\def\disleft#1:#2:#3\par{\par\hangindent#1\noindent
	 \hbox to #1{#2 \hfill \hskip .1em}\ignorespaces#3\par}

Science is what we understand well enough to explain to a computer. Art is
everything else we do. 
During the past several years an important part of mathematics
has been transformed from an Art to a Science: No longer do we need to get a
brilliant insight in order to evaluate sums of binomial coefficients, and many
similar formulas that arise frequently in practice; we can now follow a
mechanical procedure and discover the answers quite systematically.

I fell in love with these procedures as soon as I~learned them, because they
worked for me immediately. Not only did they dispose of sums that I~had
wrestled with long and hard in the past, they also knocked off two new problems
that I~was working on at the time I~first tried them. The success rate was
astonishing.

In fact, like a child with a new toy, I~can't resist mentioning how I~used the
new methods just yesterday. Long ago I~had run into the sum
$\sum_k{2n-2k\choose n-k}{2k\choose k}$, which takes the values 1, 4, 16, 64
for $n=0,1,2,3$ so it must be~$4^n$. Eventually I~learned a tricky way to prove
that it~is, indeed,~$4^n$; but if I~had known the methods in this book I~could
have proved the identity immediately. Yesterday I~was working on a harder
problem whose answer was $S_n=\sum_k{2n-2k\choose n-k}{}^2{2k\choose k}{}^2$.
I~didn't recognize any pattern in the first values 1, 8, 88, 1088, so
I~computed away with the Gosper-Zeilberger algorithm. In a few minutes
I~learned that $n^3S_n=16(n-{1\over 2})(2n^2-2n+1)S_{n-1}-256(n-1)^3S_{n-2}$. 

Notice that the algorithm doesn't just verify a conjectured identity ``$A=B$''.
It also answers the question ``What is~$A$?'', when we haven't been able to
formulate a decent conjecture. The answer in the example just considered is a
nonobvious recurrence from which it is possible to rule out any simple form
for~$S_n$. 

I'm especially pleased to see the appearance of this book, because its authors
have not only played key roles in the new developments, they also are master
expositors of mathematics. It is always a treat to read their publications,
especially when they are discussing really important stuff.

Science advances whenever an Art becomes a Science. And the state of the Art
advances too, because people always leap into new territory once they have
understood more about the old. This book will help you reach new frontiers.

\baselineskip10pt

\bigskip
\disleft 300pt::
Donald E. Knuth
\disleft 300pt::
Stanford University
\disleft 300pt::
20 May 1995

\vfill
\line{pwz.tex\hfill}

\end
