\magnification\magstep1
\font\sc=cmcsc10 %use lower case as (Monthly)
\parindent0pt
\def\bib{\par\noindent\hangindent 20pt}
\def\TeX{T\hbox{\hskip-.1667em\lower.424ex\hbox{E}\hskip-.125em X}}

\advance\baselineskip6pt
\advance\parskip6pt

\line{{\bf Knuth D E}\quad {\sl The Art of Computer
Programming}.\hfil}
\line{\qquad Reading, MA: Addison\kern.05em--Wesley.\hfil}
\line{\qquad Volume 1, {\sl Fundamental Algorithms\/} (1968) 1973, 634
p.\hfil}
\line{\qquad Volume 2, {\sl Seminumerical Algorithms\/} (1969) 1981,
688 p.\hfil}
\line{\qquad Volume 3, {\sl Sorting and Searching\/} 
1973, 723 p.\hfil}
\line{[California Institute of Technology and Stanford
University]\hfil}

\bigskip
These books are the first volumes of a projected seven-volume series
that surveys the most important techniques of computer
programming---basic methods that are being applied to the solutions of
many different kinds of problems. Each method is accompanied where
possible by a quantitative analysis of its efficiency.

\bigskip
\centerline{Artistic Programming}
\medskip
\centerline{Donald E. Knuth}
\centerline{Computer Science Department}
\centerline{Stanford University}
\centerline{Stanford, CA 94305--2140}

\bigskip
On my 24th birthday, a representative of Addison\kern.05em--Wesley
asked me whether I'd like to write a book about software creation. At
that time (1962) I~was a grad student in mathematics at Caltech. I~had
no idea that a new discipline called Computer Science would soon begin
to spring up at numerous campuses, nor did I~realize that
``deep down'' I~was really a computer scientist, not a mathematician;
computer scientists hadn't discovered each other yet. But I'd been
writing computer programs to help support my education, and the book
project was immediately appealing.

By the time I reached home that day I~had planned the book in my mind,
and I~quickly jotted down the titles of 12~chapters. But I~had almost
no time to work on the manuscript until after receiving my Ph.D. in
June 1963; I~spent the summer of~'62 writing a {\sc fortran} compiler
for UNIVAC. I~did take one day off to investigate the statistical
properties of ``linear probing,'' an important way to locate data,$^1$
and I~happened to discover a trick that made a mathematical analysis
possible. This experience profoundly changed my book-writing plans;
I~decided that a quantitative rather than qualitative approach would
be the best way to organize and present the techniques of computing.
I~also decided to emphasize aesthetics, the creation of programs that
are beautiful.$^2$

I worked feverishly during the next years and finally finished the
first draft in June, 1965. By then I~had accumulated 3000 pages of
handwritten manuscript. I~typed up Chapter~1 and sent it to the
publishers as sort of a progress report. And they said, ``Whoa, Don!
If all 12~chapters are like the first, your book will be more than
2000 pages long.'' My estimate of  manuscript pages per printed page
was off by a factor of~3.

After hectic conferences we agreed to change the original book to a
series of seven volumes. If I~had continued to type the other chapters
as they existed in 1965, all seven books would have been published
by 1970; but computer science continued its explosive growth, and
I~decided to try keeping up with current developments. Thus I~was
lucky to finish Volume~3 by 1973. By 1977 I~had completed part
of Volume~4, but the subject of that volume---combinatorial
algorithms---had become such a hot topic, more than half of all articles
in computer science journals were being devoted to it. So I~tried to
gain efficiency by taking a year off to develop computer tools for
typography. Alas, that project took 11~years.$^3$

Meanwhile people do seem to like the published volumes,$^4$ for which
I~received the Turing award in 1974 and the National Medal of Science
in 1979. The books have been translated into Russian, Chinese,
Japanese, Spanish, Romanian, and Hungarian; more than 475,000 copies
have been sold in English. They probably became ``Citation classics''
because they discuss classic principles of computing. I'm now working
full time on Volumes 4A, 4B, and~4C, which should be completed in
2003.

\bigskip\hrule\bigskip

\bib
Knuth D E. \quad Algorithms. {\sl Scientific American\/}  236:63--80,
April 1977.

\medskip
\bib
Knuth D E. \quad Computer programming as an art. {\sl Communications of the
ACM\/} 17:667--673, 1974.

\medskip
\bib
Knuth D E. \quad The errors of \TeX. {\sl Software Practice and Experience\/}
19:607--685, 1989.

\medskip
\bib
Weiss E A. \quad In the art of programming, Knuth is first; there is no
second. {\sl Abacus\/} 1(3):41--48, Spring 1984.

\bye




