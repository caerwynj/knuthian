%ershov.tex  Memories of Andrei P. Ershov by Donald E. Knuth

\magnification\magstep2
\noindent{\bf Memories of Andrei P. Ershov by Donald E. Knuth}

\medskip
The editors have asked me to note down a few of my personal memories of
Andrei Ershov. Although Andrei and~I lived on opposite sides of the globe,
with about $180^{\circ}$ degrees of longitude separating us, his life
influenced my own in many positive ways.

These influences began when I was an undergraduate student at Case Institute
of Technology. Andrei's book on his Programming Program for the BESM had just
appeared, and a bunch of us students were able to convince our Russian teacher
to include it as one of the two texts for our course on understanding
scientific Russian. This was an excellent experience for us because many of the
technical words for computer terms were not in any of our dictionaries,
nor had our teacher ever seen them before! (The English translation by
Nadler had not yet appeared.) We got a feeling that we were seeing the ``real''
Russian language as actually used in science; this was much more exciting to us
than the other text, which was about Sputnik and space exploration but at
a very simple level.

Besides learning a bit of Russian from that book, I~also learned interesting
algorithms for compiler optimization. Indeed, Andrei's early work, which
initiated this important subfield of computer science, is still of interest
today. His method of exposition also turned out to be significant:
The appealing flowchart illustrations in his book were a major influence on
the way I~later decided to illustrate program flow in my paper
``Computer-drawn flow charts'' [{\sl Communications of the ACM\/ \bf 6}
(September, 1963), 555--563], and in my subsequent series
of books on {\sl The Art of Computer Programming}.

My first personal encounter with Andrei was at an IFIP Working Group meeting
when a successor to Algol~60 was being planned. By this time I~had learned
that Andrei had been an independent co-discoverer (with Gene Amdahl)
of ``hashing with linear probing''---an important algorithm that was a
key turning point in my life because it led me to the field of algorithmic
analysis. (See the footnote on page~529 of my book {\sl Sorting and Searching\/};
this  footnote
appears on page~628 of the Russian translation.) And I~had heard
exciting rumors about new techniques incorporated in Andrei's Alpha
language project. So I~was excited to meet him in person and to learn that
he spoke English fluently. We spent about two hours talking about compilers
and languages, while he was using the Xerox machine to copy numerous
documents at that meeting.

Eventually I was able to see him more frequently, because he regularly
came to see John McCarthy at Stanford Univesity. During one of those
visits a seed was planted for one of the most memorable events of my
life, the conference on Algorithms in Modern Mathematics and Computer Science
held in Urgench, 1979. That conference---a~scientific pilgrimage to
Khowarizm, the birthplace of ``algorithm''---was a dream come true for me.
Although Andrei and~I were officially listed as co-chairmen of that meeting,
the truth is that Andrei took care of 99\% of the details, while I~was able
to relax and enjoy the proceedings and to learn important things from the many
people I~met there. 
Such an experience is a once-in-a-lifetime thing, and I~hope
it will be possible for many other computer scientists to participate
in a similar event if someone else is inspired to follow Andrei's example.
During that week I~got to know him much better than ever
before, and I~was especially struck by the brilliant way he filled numerous
roles as conference leader, organizer, philosopher, speaker, translator,
and editor. 

I have many other memories---including especially the night in 1983 when
my wife and I~took Andrei to an American square-dance party and he was
doing the Virginia Reel and ``do si do''---but the above should suffice
to explain way Andrei has had such a special significance to me personally.

On his last visit to Stanford I learned of the great work he undertook during 
the final years of his life, a~revolutionary improvement in computer science
education for millions of students; this has justly been acclaimed throughout
the world. We are all sad that Andrei's life was destined to end so 
prematurely, yet we are pleased to celebrate the many things he accomplished,
and we know that the fruits of his life will continue to nourish the
next generations of computer scientists everywhere.

\bye
